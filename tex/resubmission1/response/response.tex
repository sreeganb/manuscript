\pdfoptionpdfminorversion=4

%%%%%%%%%%%%%%%%%%%%%%%%%%%%%%%%%%%%%%%%%%%%%%%%%%%%%%%%%%%%%%%%%%%%%
%% This is a (brief) model paper using the achemso class
%% The document class accepts keyval options, which should include
%% the target journal and optionally the manuscript type.
%%%%%%%%%%%%%%%%%%%%%%%%%%%%%%%%%%%%%%%%%%%%%%%%%%%%%%%%%%%%%%%%%%%%%
\documentclass[journal=jpcbfk,manuscript=article,layout=traditional]{achemso}
%%%%%%%%%%%%%%%%%%%%%%%%%%%%%%%%%%%%%%%%%%%%%%%%%%%%%%%%%%%%%%%%%%%%%
%% Place any additional packages needed here.  Only include packages
%% which are essential, to avoid problems later.
%%%%%%%%%%%%%%%%%%%%%%%%%%%%%%%%%%%%%%%%%%%%%%%%%%%%%%%%%%%%%%%%%%%%%
\usepackage{chemformula} % Formula subscripts using \ch{}
\usepackage[T1]{fontenc} % Use modern font encodings
\usepackage{graphicx}
\usepackage{caption}
\usepackage{amsmath}
\usepackage{mathptmx}
\usepackage[scaled=0.92]{helvet}
%%%%%%%%%%%%%%%%%%%%%%%%%%%%%%%%%%%%%%%%%%%%%%%%%%%%%%%%%%%%%%%%%%%%%
%% Place any additional macros here.  Please use \newcommand* where
%% possible, and avoid layout-changing macros (which are not used
%% when typesetting).
%%%%%%%%%%%%%%%%%%%%%%%%%%%%%%%%%%%%%%%%%%%%%%%%%%%%%%%%%%%%%%%%%%%%%
\newcommand*\mycommand[1]{\texttt{\emph{#1}}}
\usepackage{microtype,xparse,tcolorbox}
\usepackage{helvet}
\newcommand{\newtext}[1]{{\color{blue} #1}}
\newcommand{\vamsee}[1]{{\color{applegreen} #1}}
\newcommand{\filipp}[1]{{\color{red} #1}}

\newenvironment{reviewer-comment }{}{}
\tcbuselibrary{skins}
\tcolorboxenvironment{reviewer-comment }{empty,
  left = 1em, top = 1ex, bottom = 1ex,
  borderline west = {2pt} {0pt} {black!20},
}
\ExplSyntaxOn
\NewDocumentEnvironment {response} { +m O{black!20} } {
  \IfValueT {#1} {
    \begin{reviewer-comment~}
      \setlength\parindent{2em}
      \noindent
      \ttfamily #1
    \end{reviewer-comment~}
  }
  \par\noindent\ignorespaces
} { \bigskip\par }

\NewDocumentCommand \Reviewer { m } {
  \section*{Comments~by~Reviewer~#1}
}
\ExplSyntaxOff
%\AtBeginDocument{\maketitle\thispagestyle{empty}\noindent}

% You can get probably get rid of these definitions:
%\newcommand\meta[1]{$\langle\hbox{#1}\rangle$}
%\newcommand\PaperTitle[1]{``\textit{#1}''}
%%%%%%%%%%%%%%%%%%%%%%%%%%%%%%%%%%%%%%%%%%%%%%%%%%%%%%%%%%%%%%%%%%%%%
\author{Sree Ganesh Balasubramani}
\affiliation{Department of Chemistry and Biochemistry, University of Arizona, 
Tucson, Arizona 85721, United States}
\author{Steven D. Schwartz}
\affiliation{Department of Chemistry and Biochemistry, University of Arizona, 
Tucson, Arizona 85721, United States}
%\author{I. Ken Groupleader}
%\altaffiliation{A shared footnote}
\email{sschwartz@email.arizona.edu}
%\phone{+123 (0)123 4445556}
%\fax{+123 (0)123 4445557}
%\affiliation[Unknown University]
%{Department of Chemistry, Unknown University, Unknown Town}
%\alsoaffiliation[Second University]
%{Department of Chemistry, Second University, Nearby Town}
%%%%%%%%%%%%%%%%%%%%%%%%%%%%%%%%%%%%%%%%%%%%%%%%%%%%%%%%%%%%%%%%%%%%%
%% The document title should be given as usual. Some journals require
%% a running title from the author: this should be supplied as an
%% optional argument to \title.
%%%%%%%%%%%%%%%%%%%%%%%%%%%%%%%%%%%%%%%%%%%%%%%%%%%%%%%%%%%%%%%%%%%%%
\title[]
  {Transition path sampling based calculations of free energies for enzymatic
  reactions: the case of human methionine adenosyl transferase and plasmodium 
  vivax adenosine deaminase}
%%%%%%%%%%%%%%%%%%%%%%%%%%%%%%%%%%%%%%%%%%%%%%%%%%%%%%%%%%%%%%%%%%%%%
%% Some journals require a list of abbreviations or keywords to be
%% supplied. These should be set up here, and will be printed after
%% the title and author information, if needed.
%%%%%%%%%%%%%%%%%%%%%%%%%%%%%%%%%%%%%%%%%%%%%%%%%%%%%%%%%%%%%%%%%%%%%
\abbreviations{TPS,WHAM}
\keywords{American Chemical Society, \LaTeX}

%%%%%%%%%%%%%%%%%%%%%%%%%%%%%%%%%%%%%%%%%%%%%%%%%%%%%%%%%%%%%%%%%%%%%
%% The manuscript does not need to include \maketitle, which is
%% executed automatically.
%%%%%%%%%%%%%%%%%%%%%%%%%%%%%%%%%%%%%%%%%%%%%%%%%%%%%%%%%%%%%%%%%%%%%
\begin{document}

%\date{\today} 

\maketitle


%Additions to the manuscript are indicated in  \textit{\newtext{blue text}}. 


\Reviewer{1}

We would like to thank the Reviewer for his/her time and for evaluating
the manuscript. 

%\begin{response}
%{The current work shows the application of the BOLAS method to two enzyme-catalyzed reactions. 
%The BOLAS method is a transition path sampling algorithm that can be extended to calculate free 
%energies along a pre-defined reaction coordinate. There are already several methods to calculate 
%free energy profiles and many have been applied to enzymatic reactions, but none is a black-box, 
%and therefore the implementation and application of this method to an enzymatic reaction is an 
%important advance. The text is well written and clear.}
%reply
%TPS based free energy calculations have been attempted previously for the calculations
%involving model problems involving analytical potential energy surfaces or for 
%configurational or structural changes of an enzyme. A chemical reaction catalyzed by 
%an enzyme can involve a complicated reaction coordinate and TPS based method using the 
%shooting algorithm provides a bias free technique for calculating the free energy
%of the reaction. 

%\end{response}

\begin{response}
{First one needs to know how the specific parameters of the method affect 
the results. Or at least, how or why they were chosen. If not for all parameters, 
at least for the most relevant ones (but this needs also to be specified). 
\newline
For example:
* why use 150 reactive trajectories?}
%reply

%The application of TPS for stuyding the kinetics and transition states of 
%enzyme catalyzed reactions has been repeorted several times by our group as well
%as by others we decided to not list out every detail. But we have now included 
%some details about the choice the number of trajectories and so on.
The number of reactive trajectories in a TPS ensemble is decided based 
on how quickly the shooting algorithm results in decorrelation of the reactive 
trajectories from the initial biased trajectory. If the decorrelation is fast 
then fewer trajectoreis are sufficient for further committor analysis. 
To be sure that we do not use trajecotries that are correlated to the initial 
biased trajectory we ignore the first 100 trajectories for committor analysis. 
Additionally we have added a discussion on the acceptance percentage of reactive 
trajectories in the manuscript. 
%It has been observed that when the acceptance 
%percentage is low the trajectories decorrelate faster.  
\end{response}

\begin{response}
  {why should each reaction last 0.5 ps? How is it known if that time is too short 
  or too long?}
We used longer trajectories (1 ps) during the initial stages of our calculations
and checked to make sure that the reaction occurs within
the 0.5 ps time length before collecting the TPS ensemble. 
\end{response}

{\begin{response}
{* why 20 windows?}
%reply
This choice was made based on the order parameter window that we are scanning.
20 windows results in a window size of 0.39 angstroms and we use a overlap of
0.09 angstroms on either side of the window to make sure that we have configurations
that are common to neighboring windows which is necessary to obtain a continuous free energy
profile. Smaller window sizes result in inefficient sampling because the TPS based method
does not include bias forces to keep the trajectories within the windows and we end up
sampling outside the windows especially in regions where the free energy has a steeper slope.  
\end{response}

\begin{response}
{* When calculating the commitor, you cannot say "repeating this procedure many times". 
How many are needed to obtain a given precision or to obtain the given figure?
\newline
* As before, when calculating the separatrix, you can not say "repeating this 
process for several trajectories".}
%reply
We have included a few lines in the committor analysis 
discussion where the details of the number of trajectories used is mentioned.  
\end{response}

\begin{response}{Some of these questions arise because nowhere in 
the manuscript are confidence intervals given. Neither in the energy 
profile nor in the transition state structure distances (Fig. 5). 
Therefore the effect of these parameters on the precision of the 
results is not known.
}
We have added a supporting information file which contains geometrical 
parameters of the transition state structures that we had calculated 
along with the mean and standard deviations. 
The transition state structures depicted in Figs. 5 and 10 are 
representatives from the ensemble. We have also updated Figs. 7 
and 11 to contain standard deviations of the free energy that were
calculated using the bootstrapping method within the windows.  
\end{response}

\begin{response}{Some of these questions also arise because the 
code to implement the method is not made public. Doing so would 
not only clarify some questions but also prevent others from 
reinventing the wheel, that is, coding the implementation again 
to check the results of this paper or to apply the method to their specific enzymes.}

% REPLY
We agree that the availability of code is necessary for reproducibility of the results.
TPS based free energies calculations are akin to vanilla TPS calculations with 
only a few modifications to the acceptance criteria as discussed in the manuscript as well
as in the original BOLAS paper. We followed the algorithm stated in the manuscript
and implemented it as python scripts but it was hard coded to model the MAT2A and 
$\textit{pv}$ADA catalyzed reactions. We would release it in a public repository once it is 
tailored for general purpose usage.  
\end{response}

\begin{response}{The second piece of information needed is how the method 
performs compared to others. Both in terms of computational cost and in 
terms of accuracy. Comparison with experiment is interesting but not very 
enlightening as the computational results depend on the method to calculate 
the free energy profile (tested here) but also on the chosen reaction coordinate, 
the QM region and the QM method, among others. A comparison with Umbrella Sampling, 
which also uses a reaction coordinate and also uses windows would be ideal. Again, 
to make fair comparisons we would need error bars to estimate the convergence of 
the results and reports of computational costs.}

We have added discussions in the manuscript and a supporting information 
document on the efficiency of sampling, 
statistical analysis of the transition state structures and free energy profiles. 
% REPLY
\end{response}

\begin{response}{Beyond these general issues raised so far, there are some other points 
that deserve some attention.
\newline
1. How is the initial reactive trajectory generated? Although obtaining an initial reactive 
trajectory is not the crux of the algorithm it is a non-trivial step that should be more 
detailed. In particular, for the pvADA enzyme, which involves a two-step reaction, how 
do the way these 2 steps take place in the initial trajectory condition of the TPS ensemble?}

% REPLY
The reaction catalyzed by the \textit{pv}ADA enzyme consists of 
the nucleophilic attack of the OH$^{-}$ on the C6 atom, proton transfer from the 
GLU 223 residue to the N1 atom in the adenosyl moeity and finally the 
the transfer of the proton from the OH$^{-}$ to the N6 atom resulting in the formation 
of NH$_3$ and inosine. We have included the time series of the key distances in 
the supporting information which shows that we collect trajectories which contain all these
steps taking the system from the reactant state to the product state with the order of events
as predicted from experiments. 
\end{response}

\begin{response}{2. Related to the previous question, considering that the barriers 
surrounding the intermediate are higher than kT, shouldn't the trajectories remain 
trapped in the intermediate? In other words, isn't the duration of the whole trajectory 
determining the mechanism by enforcing a ballistic trajectory? Is the d(NC)-d(OC) still 
a good reaction coordinate to describe the second step? If not, shouldn't that underestimate 
the barrier, something that should be compensated by a (not calculated) transmission coefficient?}

% REPLY
The calculated geometry of the intermediate resembles that of the Meissenheimer complex which is
similar to that predicted from experiments. In our TPS reactive trajectories we observe that 
once the complex is formed there is a rapid proton transfer from the nucleophilic OH$^{-}$ 
to the NH$_2$ group resulting in the release of NH$_3$ and the free energy change due to the 
proton transfer is 1.7 kCal mol$^{-1}$. 
%The order parameter that is chosen is based on the fact that
%it allows for the best definition of the reactant and product states.    
\end{response}

\begin{response}
{3. I find the free energy profiles in fig 7 and 11 worrying. Take the 4th window in 
Fig. 7: it is clearly showing a smaller slope than the interpolation and this is because 
there are jumps between the windows. The first point of this window and the next one are 
"disconnected" from the other ones. Why is this so? Repeating the calculations with smaller 
windows would show the effect of window size on the result (that would address one of my 
questions regarding the choice of parameters).}
% REPLY
Since there are no bias forces constraining the trajectories to stay within a defined window, 
in regions where the free energy curve has a large slope the trajectories tend to only sample 
configurations that are almost completely outside the window with acceptance percentages $\leq 10$.
This results in populations that are skewed in favor of regions with lower free energies. We sampled 
3000 trajectories within each window and made sure that the change in the free energy profile
is negligible as a function of number of sampled trajectories to make sure we have achieved 
convergence. Moreover the largest standard deviations that we obtained from bootstrapping 
analysis was 0.9 kCal mol$^{-1}$ which also provides a measure for convergence. 

\end{response}

\begin{response}
{4. The author should check they cite the appropriate works. I am afraid 
ref 20 is not an application of TPS to enzyme catalysis. To my knowledge, The first 
applications of TPS to enzyme catalysis are  $J. Am. Chem. Soc.2005, 127, 13822-13831$ 
and $J. Phys. Chem. B 2007, 111, 20, 5708-5718$.}
% REPLY

We thank the Reviewer for pointing out this mistake. We have made the appropriate changes to that
citation in the manuscript. 
\end{response}


\begin{response}{Overall this is a relevant work, but the authors need to go 
deeper into the scientific details to be useful to other members of the 
community, because what the manuscript should be describing is not that the 
method works --we already knew that-- but how to make it work in enzymatic reactions.}
% REPLY
\end{response}
We thank the reviewer for the constructive comments to make the manuscript better. We have 
addressed the questions by adding more statistical analysis and discussions of the details 
of the calculations. 
\section*{Other Changes}



\end{document}





